\documentclass{article}
\usepackage{amsmath,amssymb,amsfonts,amsthm}
\usepackage{verbatim}
\usepackage{listings}

\DeclareMathOperator{\fl}{fl}
\newcommand{\bangle}[1]{\langle #1 \rangle}
\newcommand{\bcurly}[1]{\{ #1 \}}
\newcommand{\TODO}[1]{\textbf{TODO : #1}}

\theoremstyle{plain}% default
\newtheorem{thm}{Theorem}[section]
\newtheorem{lem}[thm]{Lemma}
\newtheorem{prop}[thm]{Proposition}
\newtheorem*{cor}{Corollary}
\newtheorem*{KL}{Klein’s Lemma}
\theoremstyle{definition}
\newtheorem{defn}{Definition}[section]

\newtheorem{exmp}{Example}[section]
\newtheorem{xca}[exmp]{Exercise}
\theoremstyle{remark}
\newtheorem*{rem}{Remark}
\newtheorem*{note}{Note}
\newtheorem{case}{Case}

\title{Hello World!}
\author{Christian Pletbjerg}

\begin{document}
\maketitle

\section{Hello World}
Hello world! Testing some math $x = 3$... How about some more complicated math?
\begin{align}
    F( g \circ f ) 
        &=  F( g ) \circ F (f )
        \\
        &=  cool
\end{align}
And this is what verbatim outputs
\begin{lstlisting}[language=Haskell]
map :: (a -> b) -> [a] -> [b]
map _f [] = []
map f (a:as) = f a : map f as
\end{lstlisting}

asdf 

\end{document}
