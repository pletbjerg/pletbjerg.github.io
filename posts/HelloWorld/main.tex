\documentclass{article}
\usepackage{amsmath,amssymb,amsfonts,amsthm}
\usepackage{verbatim}
\usepackage{listings}
\usepackage{tikz-cd}

\DeclareMathOperator{\fl}{fl}
\newcommand{\bangle}[1]{\langle #1 \rangle}
\newcommand{\bcurly}[1]{\{ #1 \}}
\newcommand{\TODO}[1]{\textbf{TODO : #1}}

\theoremstyle{plain}% default
\newtheorem{thm}{Theorem}[section]
\newtheorem{lem}[thm]{Lemma}
\newtheorem{prop}[thm]{Proposition}
\newtheorem*{cor}{Corollary}
\newtheorem*{KL}{Klein’s Lemma}
\theoremstyle{definition}
\newtheorem{defn}{Definition}[section]

\newtheorem{exmp}{Example}[section]
\newtheorem{xca}[exmp]{Exercise}
\theoremstyle{remark}
\newtheorem*{rem}{Remark}
\newtheorem*{note}{Note}
\newtheorem{case}{Case}

\title{Hello World!}
\author{Christian Pletbjerg}

\begin{document}
\maketitle

\tableofcontents

\section{A Section}
Hello world! This is a section

\section{Testing Some Math}
Recall the following definition.
\begin{defn}
    An \emph{$\infty$-category} is a simplicial set $S_\bullet$ such that
    for every $ 0 < i < n$ and every natural transformation $\sigma_0 : \Lambda^n_i \to S_\bullet$, there exists a (not necessarily unique) map $\sigma : \Lambda^n \to S_\bullet$
    such that
    \[
        \sigma_0 = \sigma \circ \iota
    \]
    where $\iota : \Lambda^n_i \to \Lambda^n$ is the canonical injection.
\end{defn}

Also, if $F$ is a functor, we have the following straightforward statement.
\[
\begin{aligned}
    F( g \circ f ) 
        &=  F( g ) \circ F (f ) \\
        &=  F( g \circ f ) 
\end{aligned}
\]

\section{Testing Listings}
Consider the following Haskell function.
\begin{lstlisting}[language=Haskell]
map :: (a -> b) -> [a] -> [b]
map _f [] = []
map f (a:as) = f a : map f as
\end{lstlisting}

Or we can consider the following C function.
\begin{lstlisting}[language=C]
char* lex() {
    int c; 

    while ((c = getc()) != EOF && isspace(c)){;}

    switch (c) {
        case EOF:
            return "EOF";
        default:
            if (isdigit(c)) {
                do {
                    c = getc();
                } while (isdigit(c));
                ungetc(c);
                return "UINT";
            } else {
                return "ERR";
            }

    }
}
\end{lstlisting}

\end{document}
