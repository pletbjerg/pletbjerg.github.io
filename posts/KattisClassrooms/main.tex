\documentclass{article}
\usepackage{amsmath,amssymb,amsfonts,amsthm}
\usepackage{verbatim}
\usepackage{listings}
\lstset{
  basicstyle=\ttfamily,
  mathescape
}
\usepackage{tikz-cd}
\usepackage{hyperref}

\DeclareMathOperator{\fl}{fl}
\newcommand{\bangle}[1]{\langle #1 \rangle}
\newcommand{\bcurly}[1]{\{ #1 \}}
\newcommand{\TODO}[1]{\textbf{TODO : #1}}

\theoremstyle{plain}% default
\newtheorem{thm}{Theorem}[section]
\newtheorem{lem}[thm]{Lemma}
\newtheorem{prop}[thm]{Proposition}
\newtheorem*{cor}{Corollary}
\newtheorem*{KL}{Klein’s Lemma}
\theoremstyle{definition}
\newtheorem{defn}{Definition}[section]

\newtheorem{exmp}{Example}[section]
\newtheorem{xca}[exmp]{Exercise}
\theoremstyle{remark}
\newtheorem*{rem}{Remark}
\newtheorem*{note}{Note}
\newtheorem{case}{Case}

\title{Competitive Programming in Haskell: Classrooms}
\author{Christian Pletbjerg}
\date{July 25, 2023}

\begin{document}
\maketitle

\tableofcontents

\section{Introduction}
Let's say the (dreaded) club day at your school is approaching, and you've been
    (unfortunately) tasked with assigning clubs' activities at specified start
    times and finish times to a limited number of classrooms.
Clubs despise sharing so if a club is given a classroom, the club must have the
    classroom completely to themselves from its activity's start time to finish
    time (inclusive).
As there are only a limited number of classrooms, you'd like to maximize
    the number of activities that can be scheduled.

This problem is exactly the
    \href{https://open.kattis.com/problems/classrooms}{Classrooms} problem on
    Kattis which (at the time of writing this) has a difficulty of 6.9 (Hard)!
The solution presented is a (somewhat) tricky greedy algorithm.
There aren't very many Haskell solutions, so I figured I'd toss put one in
    the mix.

As an overview, we will:
\begin{itemize}
    \item Define notation.
    \item Work through an example of the problem.
    \item Design a greedy algorithm while proving its correctness.
    \item Give an implementation in Haskell.
\end{itemize}

\section{Notation}
We first fix notation for the rest of this document.

Suppose we have a nonempty set of $n$ activities $A = \{1,\dots,n\}$, 
    and a nonempty set of $m$ classrooms $C = \{1,\dots,m\}$.
Each activity $i = 1,\dots, n$ has an associated integer \emph{start time} $s_i$, and
    an associated integer \emph{finish time} $f_i$ satisfying $s_i \le f_i$ and $s_i \ge 1$ 
    i.e., activities must start before they finish, and all activities start
    at time $1$ or later.
Often, we will associate an activity $i$ with the closed interval $[s_i,f_i]$
    from its start time to finish time.
Given two distinct activities $i,j$, we say that $i,j$ \emph{overlap} (or $i$
    \emph{overlaps} with $j$) if $[s_i, f_i] \cap [s_j,f_j] \neq \emptyset$.

A \emph{schedule $\mathcal{S}$} is a function
\[
    \mathcal{S} : A \to  C \cup \{ \bot \}
\]
which \emph{assigns} each activity to a classroom or $\bot$.
Note that $\bot$ is not a classroom i.e., $\bot \not \in C$.
If $\mathcal{S}$ assigns an activity $i$ to a classroom $c \in C$, then we say
    that $i$ is \emph{scheduled}.
A schedule $\mathcal{S}$ is \emph{valid} if
    all activities in $\mathcal{S}$ that are assigned to the same classroom do not overlap  i.e.,
        all distinct activities $i,j$ assigned to the same classroom $c \in C$ (i.e., $\mathcal{S}(i) = \mathcal{S}(j) = c$)
            satisfy $[s_i,f_i] \cap [s_j,f_j] = \emptyset$.

The \emph{size} of a schedule $\mathcal{S}$ is
    the number of activities which are scheduled i.e., the size of $\mathcal{S}$ is 
\begin{equation}
    | \{ i \in A : \mathcal{S}(i) \in C \}|
    .
\end{equation}

If schedule $\mathcal{O}$ is both valid and of maximum size
    (w.r.t. valid schedules), then we say that 
    $\mathcal{O}$ is \emph{optimal}. 
Thus, in the notation given, our problem amounts to finding and optimal
    schedule.

\section{Example}
Suppose we are given $n = 5$ activities defined as follows
\begin{itemize}
    \item $s_1 = 1, f_1 = 4$
    \item $s_2 = 2, f_2 = 9$
    \item $s_3 = 4, f_3 = 7$
    \item $s_4 = 5, f_4 = 8$
    \item $s_5 = 1, f_5 = 9$
\end{itemize}
Moreover, suppose we have $m = 2$ classrooms.

One possible schedule $\mathcal{S}$ would be as follows.
\begin{align}
    1 &\mapsto 1    \\
    2 &\mapsto 2    \\
    3 &\mapsto \bot \\
    4 &\mapsto 1    \\
    5 &\mapsto \bot \\
\end{align}

We can visualize $\mathcal{S}$ as follows.
\begin{verbatim}
c1: |-----1-----|   |-----4-----|
c2:     |-------------2-------------|
    1   2   3   4   5   6   7   8   9
\end{verbatim}
\lstinline{c1} denotes classroom $1$, and \lstinline{c2} denotes classroom $2$.
A vertical bar \lstinline{|} followed by dashes \lstinline{-}, a number $x$, more dashes
    \lstinline{-}, then another vertical bar \lstinline{|}, denote activity $x$'s start and
    finish time (via the vertical bar) where we note that bottom horizontal
    numbers denote the time.

It's easy to see that this is a valid schedule,
    and this is in fact an optimal schedule of size $3$ for this instance of
    the problem.

Already, we see an interesting property of this problem.
It's clear that an optimal schedule is not
unique.
With this example here, we instead could change $\mathcal{S}$ to assign $2 \mapsto \bot$ and $5 \mapsto 2$
    which is also clearly a valid schedule and also of optimal size $3$.

\section{An Algorithm}
We propose a greedy algorithm to give us an assignment of
    activities to classrooms of maximum size. 

Suppose we are given $m > 0 $ classrooms, and $n > $ activities written as follows.
\[
    [s_1,f_1], \dots,[s_n,f_n]
\]
Without loss of generality, we may assume that the activities satisfy 
    $f_i \le f_j$ for all $i \le j$ i.e., the activities are ordered by
    earliest finish time first.


We will inductively define a schedule $\mathcal{S}$.
For every classroom $c$, we will maintain a variable $t_c$ which will 
    maintain the invariant that $t_c$ is the largest finish time of activities
    already assigned to classroom $c$ (or $0$ if no activities are assigned to classroom $c$).
Then, we define $\mathcal{S}$ as follows.
Set $t_c = 0$ for every classroom $c$.
For each activity $i$ (processed from earliest finish time to latest finish time), 
    let $c$ denote the classroom with a largest $t_c$ s.t.
        $[0,t_{c}] \cap [s_i,f_i] = \emptyset$.
Then, either:
\begin{itemize}
    \item If a classroom $c$ exists, then
            assign activity $i$ to the classroom $c$, and update $t_c$ to
            $f_i$ (this update is necessary to maintain the invariant the $t_c$
            denotes the largest finish time of activities already assigned to
            classroom $c$).
    \item Otherwise, no such classroom $c$ exists, so assign activity $i$ to
        $\bot$ i.e., activity $i$ will not be scheduled.
\end{itemize}

Obviously, we have the following properties of $\mathcal{S}$ by construction
    which can be shown by induction.
\begin{lem}
    \label{Sinvariant}
    When $\mathcal{S}$ assigns activity $\ell$ to classroom $c \in C$,
        if activity $t$ is the largest activity s.t. $t < \ell$ and $\mathcal{S}(t) = c$
            (i.e., $t$ is the activity assigned to classroom $c$ just before $\ell$),
            then for any activity $t'$ assigned to classroom $c' \in C$ with 
                $c \neq c'$ s.t. $t',\ell$ do not overlap, we have
            \begin{equation}
                f_{t'} \le f_{t}
            \end{equation}
    In other words, if $\ell$ is assigned to a classroom $c \in C$,
        then the activity assigned to classroom $c$ just before $\ell$ has the
        largest finish time of all already assigned activities that do not overlap with $\ell$.
\end{lem}
\begin{lem}
    \label{Svalid}
    $\mathcal{S}$ is valid schedule.
\end{lem}

We omit the proofs.

Since $\mathcal{S}$ is a valid schedule, to show that $\mathcal{S}$
    is an optimal schedule we just need to show that the size of $\mathcal{S}$
    is as large as possible w.r.t. valid schedules.
This is a bit tricky to show, so we'll first give an overview of the steps.
We will:
\begin{itemize}
    \item Assume that we have an optimal schedule $\mathcal{O}$.
    \item Show that we can incrementally transform $\mathcal{O}$ into a new
        optimal schedule $\mathcal{O}'$ that is more similar to
        $\mathcal{S}$.
    \item Repeatedly apply the transformation from $\mathcal{O}$ to an optimal
        schedule $\mathcal{O}'$ until we get the same schedule as $\mathcal{S}$
        which shows that $\mathcal{S}$ is optimal as well (since each
        application of the transformation is also optimal).
\end{itemize}

Suppose $\mathcal{O}$ is an optimal schedule. 
Since schedules are functions from activities to classrooms (or $\bot$),
    we may regard $\mathcal{O}$ as an assignment of the activities
    ordered by earliest finish time first.
Now, let $\ell$ denote the first activity scheduled such that $\mathcal{O}$ and
    $\mathcal{S}$ differ. 
Thus, for every activity $1 \le i < \ell$, we have that $\mathcal{O}(i) = \mathcal{S}(i)$.
We will prove the following theorem whose importance we will see shortly.
\begin{thm}
    There exists a schedule $\mathcal{O}'$ which is: 
    \begin{itemize}
        \item optimal;
        \item for every activity $j$ with $1 \le j \le \ell$, 
                we have $\mathcal{S}(j) = \mathcal{O}'(j)$ i.e.,
                    $\mathcal{O}'$ and $\mathcal{S}$ schedule activity $j$ the same
                way.
    \end{itemize}
\end{thm}
\begin{proof}
    We will describe how to construct the schedule $\mathcal{O}'$.
    Then, to argue that $\mathcal{O}'$ is optimal,
        we will show that: $\mathcal{O}'$ does not assign an activity to a
            classroom s.t. an activity will overlap with a distinct activity
            already assigned to the same classroom; 
            and $\mathcal{O}'$ is the same size as $\mathcal{O}$ which is
            already known to be of maximum size w.r.t. valid schedules as we
            assumed $\mathcal{O}$ is optimal.

    For every activity $1 \le i < \ell$, 
        put 
        \begin{equation}
            \mathcal{O}'(i) \mapsto \mathcal{O}(i) = \mathcal{S}(i) 
        \end{equation}
        i.e., assign each activity $i$ the same way $\mathcal{O}$ does (which
        we may recall that by defn. of $\ell$, this is the same as $\mathcal{S}$).
    Then, for activity $\ell$, put 
        \begin{equation}
            \mathcal{O}'(\ell) \mapsto \mathcal{S}(\ell)
        \end{equation}
        i.e., assign activity $\ell$ the same way $\mathcal{S}$ does.

    For the remaining activities $k > \ell$, we must distinguish cases on how $\mathcal{S}$
        and $\mathcal{O}$ assign activity $\ell$.
    \begin{itemize}
        \item If $\mathcal{S}(\ell) = \bot$ and $\mathcal{O}(\ell) = c \in C$,
                then this means that $\mathcal{S}$ was unable to assign activity $\ell$ to a
                classroom so the largest finish time of every activity assigned
                to classroom thus far overlaps with activity $\ell$.
                But, since $\mathcal{S}(i) = \mathcal{O}(i)$ for every activity $1 \le i < \ell$,
                    this would mean that since $\mathcal{O}$ is optimal and hence a valid schedule,
                    we get that there must exist a classroom which does not overlap with activity $\ell$
                    -- contradicting that $\mathcal{S}$ was unable to assign activity $\ell$ to a classroom.
                Hence, this case is impossible.

        \item Suppose $\mathcal{S}(\ell) = c \in C$ and $\mathcal{O}(\ell) = \bot$.
            Then, for every activity $k > \ell$, we put
            \begin{equation}
                \mathcal{O}'(k) \mapsto
                    \begin{cases}
                        \bot & \text{if $\ell, k$ overlap and $\mathcal{O}(k) = c$} \\
                        \mathcal{O}(k) & \text{otherwise} \\
                    \end{cases}
            \end{equation}

            Observe that the only activities $\mathcal{O}$ and $\mathcal{O}'$
                assign differently are:
            \begin{itemize}
                \item activity $\ell$ where we recall $\mathcal{O}'(\ell) = \mathcal{S}(\ell) =c$; and
                \item any activity $k > \ell$ for which $\mathcal{O}(k) =c$ and
                    $k,\ell$ overlap where we put $\mathcal{O}'(k) = \bot$.
            \end{itemize}

            Obviously, this definition makes $\mathcal{O}'$ a valid schedule
                since $\mathcal{O}$ is a valid schedule, and any activity which
                could overlap with another activity assigned to the same
                classroom in $\mathcal{O}'$ must then be assigned to classroom
                $c$ and overlap with activity $\ell$, but such an activity is
                not scheduled in $\mathcal{O}'$.
            Thus, $\mathcal{O}'$ is a valid schedule.

            Now, to show that $\mathcal{O}'$ is optimal, all that remains is to
                show that $\mathcal{O}'$ has the largest size w.r.t. valid schedules.
            We show this by showing that $\mathcal{O}'$ has the same size as
                $\mathcal{O}$.
            Note that $\mathcal{O}'$ assigns $\ell$ to a classroom, but
                $\mathcal{O}$ does not assign $\ell$ to a classroom.
            So, to show that $\mathcal{O}'$ is optimal, since 
                $\mathcal{O}'$ is essentially identical to $\mathcal{O}$,
                it suffices to show that there is at most one activity $k' > \ell$
                which satisfies $\ell,k'$ overlap, and $\mathcal{O}(k') = c$.
            Suppose for the sake of contradiction that $k',k'' > \ell$ are distinct,
                both overlaps with $\ell$, and $\mathcal{O}(k') = \mathcal{O}(k'') = c$.
            Without loss of generality, we may assume that $k' < k''$.
            Using that we assumed all activities are ordered by earliest finish time first, we know that
            \begin{equation}
                f_{\ell} \le f_{k'} \le f_{k''}
            \end{equation}
            So, using that we assumed that both $k',k''$ overlap with $\ell$, 
                it's easy to see that
                $ f_\ell \in [s_\ell, f_\ell] \cap [s_{k'}, f_{k'}]$
                and
                $f_\ell \in [s_\ell, f_\ell] \cap [s_{k''}, f_{k''}]$.
            Thus, $f_\ell \in [s_{k'}, f_{k'}] \cap [s_{k''}, f_{k''}]$ follows i.e.,
                $k',k''$ overlap and are assigned to the same classroom
                $c$ in $\mathcal{O}$ -- contradicting that $\mathcal{O}$ is valid.

            %%%%%%%%%%%%%%%%%%%%%%%%%%%%%%%%%%%%%%%%%%%%
            \item Suppose $\mathcal{S}(\ell) = c \in C$
                and $\mathcal{O}(\ell) = c' \in C$ where we necessarily have $c
                \neq c'$ since $\ell$ is the first activity for which
                $\mathcal{S}$ and $\mathcal{O}$ differ.

            Then, for every activity $k > i$, we put
            \begin{equation}
                \mathcal{O}'(k) \mapsto
                    \begin{cases}
                        c' & \text{if $\mathcal{O}(k) = c$} \\
                        c & \text{if $\mathcal{O}(k) = c'$} \\
                        \mathcal{O}(k) & \text{otherwise}
                    \end{cases}
            \end{equation}

            Observe that the only activities $\mathcal{O}$ and $\mathcal{O}'$
                assign differently are:
            \begin{itemize}
                \item activity $\ell$ where we recall $\mathcal{O}'(\ell) = \mathcal{S}(\ell) =c \neq c' = \mathcal{O}(\ell)$; and
                \item any activity $k > \ell$ for which either 
                    $\mathcal{O}(k) = c$ (so $\mathcal{O}'(k) = c'$)
                    or $\mathcal{O}(k) =c'$ (so $\mathcal{O}'(k) = c$).
            \end{itemize}

            It's clear that $\mathcal{O}'$ has the same size as $\mathcal{O}$,
                and hence has the largest size w.r.t. valid schedules.
            So, to show that $\mathcal{O}'$ is optimal, we need to show that
                $\mathcal{O}'$ is valid i.e., all activities assigned to the
                same classrooms do not overlap.
            Since $\mathcal{O}$ and $\mathcal{O}'$ are essentially identical,
                we only need to show that the activities assigned to classrooms
                $c$ (and $c'$ resp.)
                do not overlap.

            We first show that no activities assigned to classroom $c$
                in $\mathcal{O}'$ overlap.
            Note that $\ell$ since $\mathcal{O}'(\ell) = \mathcal{S} = c$,
                by defn. of $\mathcal{S}$,
                    we know that $\ell$ does not overlap with any activity 
                    $1 \le i < \ell$ assigned to $c$ in $\mathcal{O}'$.
            Recall that $\mathcal{O}(\ell) = c'$, so for any activity $k > \ell$ 
                satisfying $\mathcal{O}'(k) = c$, we know that $\mathcal{O}(k) = c'$,
                so since $\mathcal{O}$ is valid, this implies that no activity
                $k$ can overlap with any activity previously assigned to $c$ in
                $\mathcal{O}'$.

            Now, we show that no activities assigned to classroom $c'$
                in $\mathcal{O}'$ overlap.
            Obviously, all activities $1 \le i < \ell$ assigned to classroom
                $c'$ in $\mathcal{O}'$ do not overlap.

            Let $t < \ell$ denote the largest activity satisfying 
                $\mathcal{O}'(t) = c$,
                and let $t' < \ell$ denote the largest activity satisfying
                $\mathcal{O}'(t') = c'$.
            Note that $\ell \ge \ell$ is the smallest activity satisfying
                $\mathcal{O}'(\ell) = c$,
                and let $\ell' > \ell$ be the smallest activity satisfying
                $\mathcal{O}'(\ell') = c'$ (so $ \mathcal{O}(\ell') = c$).
            If such activities do not exist, then the following results
                are trivial.

            We show that $t',\ell'$ do not overlap.
            By defn. of $\mathcal{S}$, since $\mathcal{S}(\ell) = c$,
                we get that
            \begin{equation}
                f_{t'} \le f_{t}
            \end{equation}
                since
                    $t'$ and $\ell$ do not overlap (since $\mathcal{O}(t') = c'$, $\mathcal{O}(\ell) = c'$, and $\mathcal{O}$ is valid)
                    and $\mathcal{S}$ assigned activity $\ell$ to the classroom 
                    with the largest $f_t$ s.t. $\ell,t$ do not overlap.

            Then, since we know $\mathcal{O}(t) = c$, $\mathcal{O}(\ell') = c$, and $\mathcal{O}$
                is valid, we get that $t$ and $\ell'$ do not overlap,
                    so since $f_{t'} \le f_{t}$, it certainly follows that
                        $t'$ and $\ell'$ do not overlap as required.

            Finally, for any $k > \ell' > \ell$ satisfying $\mathcal{O}'(k) = c'$,
                we know that $\mathcal{O}(k) = c$, so since $\mathcal{O}$ is
                valid, this implies that no activity $k$ can overlap with any
                activity previously assigned to $c'$ in $\mathcal{O}'$.

    \end{itemize} 
    It's easy to see that this concludes all possible cases, so the proof is complete.

    For completeness, we include full construction of $\mathcal{O}'$ omitting the impossible cases.
    \begin{equation}
        \mathcal{O}'(i) \mapsto
            \begin{cases}
                \mathcal{O}(i) = \mathcal{S}(i) & \text{if $1 \le i < \ell$} \\
                \mathcal{S}(i) & \text{if $i = \ell$} \\
                \begin{cases} 
                    \begin{cases} 
                        \bot & \text{if $\ell,i$ overlap and $\mathcal{O}(k) = c$} \\
                        \mathcal{O}(i) & \text{otherwise} \\
                    \end{cases} 
                            & \text{if $\mathcal{S}(\ell) = c \in C$ and $\mathcal{O}(\ell) =\bot$} \\
                    \begin{cases} 
                        c' & \text{if $\mathcal{O}(i) = c$} \\
                        c & \text{if $\mathcal{O}(i) = c'$} \\
                        \mathcal{O}(i) & \text{otherwise} \\
                    \end{cases} 
                            & \text{if $\mathcal{S}(\ell) = c \in C$ and $\mathcal{O}(\ell) =c' \in C$} \\
                \end{cases} 
                    & \text{if $i > \ell$} \\
            \end{cases}
    \end{equation}
\end{proof}
Finally, we may apply this theorem at most $n$ times (since there are at most
    $n$ activities in any schedule) to make $\mathcal{O}$ 
    identical to $\mathcal{S}$ while still being optimal.
This immediately proves correctness of our algorithm and is summed up with the
    following theorem.
\begin{thm}
    $\mathcal{S}$ is an optimal solution.
\end{thm}

\section{An Implementation in Haskell}
If we made it through that rather wordy proof, the implementation in
    Haskell shouldn't be that tricky.
We need to do two things:
\begin{enumerate}
    \item Sort the input activities by earliest finish time.
    \item Maintain an efficient way find the classroom with the largest 
        $t_c$ (i.e., the largest finish time of all already assigned
        activities for each classroom) that does not overlap with the next activity's start time.
\end{enumerate}
We know that we can sort lists in time $O(n \log n)$. 
Moreover, we can use an ordered
    tree of size $m$ containing the variables $t_c$ (i.e., the largest finish
    time of all already assigned activities for each classroom) to find the
    largest $t_c$ which does not overlap with the activity currently being
    processed in time $O(\log m)$.
We must repeat this operation $n$ times, so this will time $O(n \log m)$.
Altogether, this algorithm will take time $O(n \log n + n \log m)$.

Note that in the Kattis question, 
    it suffices to just return the size of the optimal solution and it is not
    necessary to actually compute a schedule.
Thus, when implementing the ordered tree containing the variables $t_c$, 
    it is not necessary to do the extra book keeping of the associated
    classroom for $t_c$, and we may instead just maintain a multiset of
    variables $t_c$.

Luckily, there are some libraries on the Kattis server which provides us with
these mechanisms.
In particular, \lstinline{Data.List.sortOn} can sort a list in $O(n \log n)$,
    and we may use an \lstinline{Data.IntMap.Strict.IntMap} to implement a
    multiset that can be used to query the largest $t_c$ which does not overlap
    with the next activity's start time in $O(\log m)$ with the function
    \lstinline{Data.IntMap.Strict.lookupLT}.

Concretely, an implementation is as follows.
\lstinputlisting[language=Haskell]{./Classrooms.hs}

\end{document}
